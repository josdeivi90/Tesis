% ------------------------------------------------------------------------
% ------------------------------------------------------------------------
% ------------------------------------------------------------------------
%                                Abstract
% ------------------------------------------------------------------------
% ------------------------------------------------------------------------
% ------------------------------------------------------------------------
\chapter*{ABSTRACT}

\footnotesize{
\begin{description}
  \item[TITLE:] OPTIMAL CONTROL OF A MICROGRID FROM A HIERARCHICAL VIEWPOINT\footnote{Bachelor Thesis}
  \item[AUTHOR:] SEBASTI�N BENJUMEA CERPA\footnote{Facultad de Ingenier�as F�sico-Mec�nicas. Escuela de Ingenier�as El�ctrica, Electr�nica y Telecomunicaciones. Director: Ricardo Alzate Casta�o, Doctorado en Ingenier�a Inform�tica y Autom�tica.}
  \item[KEYWORDS:] ELECTRICAL MICROGRID, ENERGY RESOURCE MANAGEMENT, HIERARCHICAL CONTROL, RENEWABLE ENERGY.
  \item[DESCRIPTION:]\hfill \\ In this work, the design and implementation through numerical simulation of an energy management strategy for resources of an isolated DC microgrid employing renewable sources, is performed. First of all, a review about concepts regarding the control of DC electronic power circuits of the boost type and the scheduling of power in generation schemes via the so-called droop control technique, are covered in order to configure a hierarchical structure of power management aiming at implementing economic dispatch of resources attending cost requirements for renewable sources, including as part of a microgrid scheme configured to supply the power demand of a charging station for electric vehicles with a nominal load of 1500 W. Accordingly, numerical simulations in PSIM were performed to verify the hierarchical control scheme achieving dynamical scheduling of generation proportions (power management) by modifying the droop parameter subjected to a cost functional suited for renewable sources. Moreover, the implementation of a numerical Newton-Raphson algorithm was performed in a C-code block of PSIM to approximate the optimal solution, reducing the operation cost of the microgrid. Ongoing work include the extension of presented results to the case of alternating current generation systems and the experimental verification for the dispatch algorithm proposed.
\end{description}}\normalsize
% ------------------------------------------------------------------------ 