% ------------------------------------------------------------------------
% ------------------------------------------------------------------------
% ------------------------------------------------------------------------
%                                Resumen
% ------------------------------------------------------------------------
% ------------------------------------------------------------------------
% ------------------------------------------------------------------------
\chapter*{RESUMEN}

\footnotesize{
\begin{description}
  \item[T�TULO:] CONTROL �PTIMO DE MICRORREDES A PARTIR DE UN ENFOQUE JER�RQUICO \footnote{Trabajo de grado}
  \item[AUTOR:] SEBASTI�N BENJUMEA CERPA \footnote{Facultad de Ingenier�as F�sico-Mec�nicas. Escuela de Ingenier�as El�ctrica, Electr�nica y Telecomunicaciones. Director: Ricardo Alzate Casta�o, Doctorado en Ingenier�a Inform�tica y Autom�tica.}
  \item[PALABRAS CLAVE:] CONTROL JER�RQUICO, ENERG�AS RENOVABLES, GESTI�N DE RECURSOS ENERG�TICOS, MICRORREDES EL�CTRICAS.
  \item[DESCRIPCI�N:]\hfill \\ El presente trabajo de grado presenta el dise�o e implementaci�n (a trav�s de simulaci�n) de una estrategia para la gesti�n optimizada de recursos energ�ticos en una microrred DC aislada, que hace uso de fuentes renovables. Inicialmente, el trabajo aborda una revisi�n conceptual acerca del control de circuitos convertidores de potencia del tipo elevador (boost) y se complementa por el problema de reparto de potencias a trav�s de esquemas de control droop. Posteriormente, se establece una estructura jer�rquica para gobernar la asignaci�n de proporciones de generaci�n (despacho) en una microrred, atendiendo a criterios de costo definidos considerando recursos renovables. A partir de ello, se toma como caso de estudio una estaci�n de carga para autom�viles el�ctricos, dimensionada para operar a una carga nominal de 1500 W, y a la cual se aplica la estrategia de control jer�rquico utilizando simulaciones en PSIM. Para obtener la asignaci�n din�mica (gesti�n) de recursos en la microrred se realiz� la codificaci�n del algoritmo de Newton-Raphson, como m�todo de aproximaci�n para determinar la soluci�n �ptima del problema y por ende, la reducci�n del costo de operaci�n de la microrred. Trabajos futuros incluyen la extensi�n de los resultados presentados al caso de sistemas de generaci�n en corriente alterna y la verificaci�n experimental para el algoritmo de gesti�n de recursos propuesto.
\end{description}}\normalsize
% ------------------------------------------------------------------------ 